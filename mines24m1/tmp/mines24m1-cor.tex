\parti{Calcul d'une intégrale}
\xsol
    Soit $\theta\in\ii]-\pi,\pi[$. La fonction $t\longmapsto 1+t\e^{\i\theta}$ ne s'annule pas sur $\ii ]0,+\infty[$ donc $f$ est continue sur $\ii]0,+\infty[$. En outre
    \<\al{}
        f(t) & \sim_{t\to 0}\frac1{t^{1-x}} &&&
        |f(t)| & \sim_{t\to+\infty}\frac{1}{t^{2-x}}
    \>
    avec $1-x<1$ et $2-x>1$. Donc par comparaison à des fonctions de Riemann,
    \<\r
        la fonction $f$ est intégrable sur $\ii]0,+\infty[$
    \>
\xsol
    Considérons la fonction
    \<
        \func*{\varphi}{(\theta,t)}{\frac{t^{x-1}}{1+t\e^{\i\theta}}} \quad (\theta,t)\in D=\ii ]-\pi,\pi[\times \ii]0,+\infty[
    \>
    \begin{discussion}
    \unite
    $\varphi$ est de classe $\mathcal C^1$ sur $D$ et on a
    \<
        \xderp \varphi\theta(\theta, t)=-\frac{\i t^x\e^{\i \theta}}{(1+t\e^{\i \theta})^2}
    \>
    \unite
    La fonction $r$ est bien définie sur $\ii]-\pi,\pi[$ selon la question précédente
    \unite Soit comme suggéré par l'indication de l'énoncé $\beta\in\ii]0,\pi[$. Fixons $t>0$
    \<
        \delim\sz2| 1+t\e^{\i \theta}|^2 =
        \delim\sz2(1+t\cos\theta)^2+t^2\sin^2\theta  =
        1+2t\cos\theta+t^2
    \>
    La fonction $\cos$ est décroissante sur $\ii[0,\pi]$ donc
    \<
        \xforall \theta\in\ii[0,\beta]\;
        1+2\cos\theta+t^2\geq 1+2\cos\beta+t^2
    \>
    ce qui amène, par parité de la fonction $\cos$
    \<
        \xforall \theta\in\ii[-\beta,\beta]\;
        \delim| 1+t\e^{\i \theta}|^2\geq \delim| 1+t\e^{\i \beta}|^2
    \>
    \<\lt{On en déduit que}
        \xforall (\theta,t)\in\ii[-\beta,\beta]{}\times{}]0,+\infty[\;
        \delim|\xder \varphi\theta(\theta,t)|\leq \frac{t^x}{|1+t\e^{\i \beta}|^2}
    \>
    La fonction $\rho=\delim|\xder \varphi\theta(\beta,\cdot)|$ est continue et elle est intégrable sur $\ii]0,+\infty[$ car
    \<\al{}
        \delim\sz4|\xder\varphi\theta(\beta,t)| &\sim_{t\to0} t^x &&&
        \delim\sz4|\xder\varphi\theta(\beta,t)| &\sim_{t\to+\infty} \frac1{t^{2-x}}
    \>
    avec $x>0$ et $2-x>1$.
    \end{discussion}
    Toutes les hypothèses sont réunies pour pouvoir appliquer la formule de Leibniz sur l'intervalle $[-\beta,\beta]$. Le réel $\beta$ étant quelconque dans $\ii]0,\pi[$ on conclut que
    \<\r\wd{100}
        La fonction $r$ est de classe $\mathcal C^1$ sur $\ii]-\pi,\pi[$ et
        \<
            \xforall \theta\in\ii]-\pi,\pi[\;
            r'(\theta)=
            -i\e^{\i \theta} \int_0^{+\infty} \frac{t^x}{\left(1+t \mathrm{e}^{\i \theta}\right)^2} \diff t
        \>
    \>
\xsol
    Notons que pour tout $\theta\in\ii]-\pi,\pi[$
    \< g(\theta)=\e^{\i x\theta}r(\theta) \>
     La fonction $g$ est de classe $\mathcal C^1$ comme produit de deux fonctions qui le sont et on a
    \<\al{}
        g'(\theta) &=
        \i x\e^{\i x\theta}r(x)+\e^{\i x\theta}r'(\theta) \\ &=
        i\e^{\i x\theta}\delim(\xint_0^{+\infty}[x\frac{t^{x-1}}{1+t\e^{\i\theta}}-\e^{i\theta}\frac{t^x}{(1+t\e^{\i\theta})^2}]) \\ &=
        \i\e^{\i x\theta}\xint_0^{+\infty}{h'(t)}
    \>
    où $h$ est la fonction indiquée dans l'énoncé :
    \< h(t)=\frac{t^x}{1+t \e^{\i\theta}} \>
    On a $h(t)\lsim_{t\to 0}t^x$, $h(t)\lsim_{t\to\infty}\xfrac{{\e^{-\i\theta}}}{{t^{1-x}}}$ et $x\in\ii]0,1[$
    \<\al{}
        \llim_{t\to0}h(t) &= 0 &&& \llim_{t\to+\infty}h(t)=0
    \>
    On en déduit que
    \<
        g'(\theta)=\i\e^{\i\theta}\delim(\lim_{t\to+\infty}h(t)-\lim_{t\to 0}h(t))=0
    \>
    $g$ est de classe $\mathcal C^1$ de dérivée nulle sur \emph{l'intervalle} $\ii]-\pi,\pi[$ donc
    \<\r
        La fonction $g$ est constante sur $\ii]-\pi,\pi[$.
    \>
    \begin{nb}
        Une simple intégration par partie effectuée sur l'intégrale dans $r'(\theta)$ aboutit à la relation
        \< r'(\theta)+\i xr(\theta)=0 \>
        exprimant ainsi que $g'(\theta)=0$.
    \end{nb}
\xsol
    Soit $\theta\in\ii]0,\pi[$. La fonction $g$ est constante et on a $\OV{g(\theta)}=g(-\theta)=g(\theta)$ donc
    \begin{xalign}
        \<
            g(\theta)\sin(x\theta) &=
            \frac1{2\i}\delim(g(\theta)\e^{\i x\theta}-g(\theta)\e^{-\i x\theta})
        \>
        \eline
        \<\lt{donc}\fr{result}
           \sff &=
            \frac1{2\i}\delim\sz2(g(-\theta)\e^{\i x\theta}-g(\theta)\e^{-\i x\theta})
        \>
        \eline
        \<
            &=
            \xim\sz2(g(\theta)\e^{-\i x\theta})
        \>
        (car $g(-\theta)=\OV{g(\theta)}$)
        \eline
        \<
            &=
            \xim(\int_0^{+\infty}\frac{t^{x-1}}{1+t\e^{\i \theta}}\diff t)
        \>
        \eline
        \<
            &=
            \xim(\int_0^{+\infty}\frac{t^{x-1}(1+t\e^{-i\theta})}{|1+t\e^{\i \theta}|^2}\diff t)
        \>
        \eline
        \<\lt{d'où}\fr{result}
            \sff &=
            \sin(\theta)\int_0^{+\infty}\frac{t^x}{1+2t\cos\theta+t^2}\diff t
        \>
    \end{xalign}

\xsol
Soit $\theta\in\ii]0,\pi[$. Sachant que $\sin\theta\ne0$ on peut poser pour tout réel $t>0$, $t=u\sin\theta-\cos\theta$. On a alors
    \<t^2+2t\cos\theta+1=(t+\cos\theta)^2+\sin^2\theta=(1+u^2)\sin^2\theta\>
L'application $u\longmapsto u\sin\theta-\cos\theta$ est une bijection de classe $\mathcal C^1$ de $\ii]\cotan\theta,+\infty[$ sur $\ii]0,+\infty[$ et l'integrale $\int_{0}^{+\infty}\frac{t^x}{1+2t\cos\theta+t^2}\diff t$ est convergente donc le changement de variable $t=u\sin\theta-\cos\theta$ donne
\<
\int_{0}^{+\infty}\frac{t^x}{1+2t\cos\theta+t^2}\diff t=
\frac1{\sin\theta}\int_{\xcotan \theta}^{+\infty} \frac{(u \sin (\theta)-\cos (\theta))^x}{1+u^2} \diff  u
\>
\<\lt{Soit}\fr{result}
    g(\theta) \sin (\theta x)=\int_{\xcotan \theta}^{+\infty} \frac{(u \sin (\theta)-\cos (\theta))^x}{1+u^2} \diff  u
\>

\xsol
    Dans le but d'utiliser le \ac{tcd}, considérons une suite $(\theta_n)_n$ d'éléments de $\ii[0,\pi[$ qui converge ves $\pi$. On a alors pour tout $n\in\N$
    \<
        g(\theta_n)\sin(x\theta_n)=
        \int_{0}^{+\infty}\frac{t^x\sin(\theta_n)}{1+2t\cos\theta_n+t^2}\diff t=
        \int_{0}^{+\infty}h_n(t)\diff t
    \>
\parti{Une expression (utile) de la fonction sinus}
\parti{Calcul d'une intégrale de Dirichlet généralisée}
\parti{Calcul de \protect \(E(|S_n|)\protect \)}
