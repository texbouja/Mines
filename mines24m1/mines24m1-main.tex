\documentclass[11pt,solution]{cpgedev}

\cpgegeometry{tablet}
\cpgefont{utopia}

\usepackage{parskip}
\Document{Sujet de concours}
\Concours{Mines et Ponts}
\Epreuve{Mathématique 1}
\Session{2024}   

 
\begin{document} 

\cpgesetuplists{
    enumi={n=|1|, d*=\bfseries#1, m*},
}
\cpgesetupsectiontitle{parti={
            label=|I|,
            label deco=\bfseries Partie #1~:~,
            fill=right
}} 
\cpgesetupmath{symbol={C=\mathbf C, R=\mathbf R, N=\mathbf N}}
\def\i{\mathrm i}

\titre( 
    |document| \par
    \iftitlepage\bigskip\hrule\bigskip\fi
    |concours| \\
    |session|  \\
    |epreuve|  \\
    |auteur|)

\tableofcontents  

\begin{enonce}{mines24m1}\op{corrige={before title=\clearpage}}
    (Généralisation d'une intégrale de Dirichlet et application)

    Le but de ce sujet est de calculer l'intégrale de Dirichlet généralisée
\<
    \int_0^{+\infty} \frac{1-(\cos (t))^{2 p+1}}{t^2} \diff  t
\>
    et d'utiliser ce calcul pour évaluer une espérance.

\wparti{Calcul d'une intégrale}

    Dans tout ce qui suit, $x$ est un élément de $\ii ] 0 , 1[$ fixé.
\xques+ %1 
    Montrer que pour tout $\theta \in\ii ]-\pi , \pi[$, la fonction $f$ définie par
    \<
        \func f{\ii] 0,+\infty[}  \C  
        t {\ds\frac{t^{x-1}}{1+t \e^{\i \theta}}}
    \>
    est définie et intégrable sur $\ii ] 0 ,+\infty[$.
\begin{solution}
    Soit $\theta\in\ii]-\pi,\pi[$. La fonction $t\longmapsto 1+t\e^{\i\theta}$ ne s'annule pas sur $\ii ]0,+\infty[$ donc $f$ est continue sur $\ii]0,+\infty[$. En outre
    \<\al{}
        f(t) & \sim_{t\to 0}\frac1{t^{1-x}} &&&
        |f(t)| & \sim_{t\to+\infty}\frac{1}{t^{2-x}}
    \>
    avec $1-x<1$ et $2-x>1$. Donc par comparaison à des fonctions de Riemann, 
    \<\r 
        la fonction $f$ est intégrable sur $\ii]0,+\infty[$
    \>
\end{solution}
\exit 
    Soit $r$ la fonction définie par
    \<
        \func r {\ii ] -\pi , \pi[}   \C  
        \theta {\ds \int_0^{+\infty} \frac{t^{x-1}}{1+t \e^{\i \theta}} \diff  t }
    \>

\xques\r %2
    Montrer que la fonction $r$ est de classe $\mathcal{C}^1$ sur $\ii ]-\pi , \pi[$ et que :
    \<
        \forall \theta \in\ii ]-\pi , \pi[, \quad r^{\prime}(\theta)=-\i \e^{\i \theta} \cdot \int_0^{+\infty} \frac{t^x}{\left(1+t \mathrm{e}^{\i \theta}\right)^2} \diff  t .
    \>

    \begin{ind}
        soit $\beta \in\ii] 0 , \pi[$, montrer que pour tout $\theta \in\ii [-\beta , \beta]$ et $t \in[0,+\infty[$,
    \<
        \left|1+t \e^{\i \theta}\right|^2 \geq\left|1+t \e^{\i \beta}\right|^2=(t+\cos (\beta))^2+(\sin (\beta))^2
    \>
    \end{ind}

\begin{solution}
    Considérons la fonction 
    \<
        \func*{\varphi}{(\theta,t)}{\frac{t^{x-1}}{1+t\e^{\i\theta}}} \quad (\theta,t)\in D=\ii ]-\pi,\pi[\times \ii]0,+\infty[ 
    \>
    \begin{discussion}
    \unite 
    $\varphi$ est de classe $\mathcal C^1$ sur $D$ et on a 
    \<
        \xderp \varphi\theta(\theta, t)=-\frac{\i t^x\e^{\i \theta}}{(1+t\e^{\i \theta})^2}
    \>
    \unite 
    La fonction $r$ est bien définie sur $\ii]-\pi,\pi[$ selon la question précédente
    \unite Soit comme suggéré par l'indication de l'énoncé $\beta\in\ii]0,\pi[$. Fixons $t>0$ 
    \< 
        \delim\sz2| 1+t\e^{\i \theta}|^2 = 
        \delim\sz2(1+t\cos\theta)^2+t^2\sin^2\theta  =
        1+2t\cos\theta+t^2
    \>
    La fonction $\cos$ est décroissante sur $\ii[0,\pi]$ donc 
    \<
        \xforall \theta\in\ii[0,\beta]\; 
        1+2\cos\theta+t^2\geq 1+2\cos\beta+t^2
    \>
    ce qui amène, par parité de la fonction $\cos$
    \< 
        \xforall \theta\in\ii[-\beta,\beta]\;
        \delim| 1+t\e^{\i \theta}|^2\geq \delim| 1+t\e^{\i \beta}|^2
    \>
    \<\lt{On en déduit que}
        \xforall (\theta,t)\in\ii[-\beta,\beta]{}\times{}]0,+\infty[\;
        \delim|\xder \varphi\theta(\theta,t)|\leq \frac{t^x}{|1+t\e^{\i \beta}|^2}
    \>
    La fonction $\rho=\delim|\xder \varphi\theta(\beta,\cdot)|$ est continue et elle est intégrable sur $\ii]0,+\infty[$ car 
    \<\al{}
        \delim\sz4|\xder\varphi\theta(\beta,t)| &\sim_{t\to0} t^x &&&
        \delim\sz4|\xder\varphi\theta(\beta,t)| &\sim_{t\to+\infty} \frac1{t^{2-x}}
    \>
    avec $x>0$ et $2-x>1$.
    \end{discussion}
    Toutes les hypothèses sont réunies pour pouvoir appliquer la formule de Leibniz sur l'intervalle $[-\beta,\beta]$. Le réel $\beta$ étant quelconque dans $\ii]0,\pi[$ on conclut que 
    \<\r\wd{100}
        La fonction $r$ est de classe $\mathcal C^1$ sur $\ii]-\pi,\pi[$ et 
        \<  
            \xforall \theta\in\ii]-\pi,\pi[\;
            r'(\theta)=
            -i\e^{\i \theta} \int_0^{+\infty} \frac{t^x}{\left(1+t \mathrm{e}^{\i \theta}\right)^2} \diff t 
        \>
    \>
\end{solution} 
 

 
Soit $g$ la fonction définie par
\<
    \func g{\ii ]-\pi , \pi[} \C
    \theta {\ds \e^{\i x \theta} \int_0^{+\infty} \frac{t^{x-1}}{1+t \e^{\i \theta}} \diff  t} 
\>

\xques %3
    Montrer que la fonction $g$ est de classe $\mathcal{C}^1$ sur $\ii ]-\pi , \pi[$ et que pour tout $\theta \in\ii ]-\pi , \pi[$,
    \<
        g^{\prime}(\theta)=\i \e^{\i x \theta} \int_0^{+\infty} h^{\prime}(t) \diff  t
    \>
    où $h$ est la fonction définie par
    \<
        \func h{\ii]0,+\infty[} \C  
        t {\ds\frac{t^x}{1+t \e^{\i\theta}}} 
    \>
Calculer $h(0)$ et $\llim _{t \to+\infty} h(t)$.
En déduire que la fonction $g$ est constante sur $\ii ]-\pi,\pi[$.

\begin{solution}
    Notons que pour tout $\theta\in\ii]-\pi,\pi[$
    \< g(\theta)=\e^{\i x\theta}r(\theta) \>
     La fonction $g$ est de classe $\mathcal C^1$ comme produit de deux fonctions qui le sont et on a 
    \<\al{}
        g'(\theta) &= 
        \i x\e^{\i x\theta}r(x)+\e^{\i x\theta}r'(\theta) \\ &=
        i\e^{\i x\theta}\delim(\xint_0^{+\infty}[x\frac{t^{x-1}}{1+t\e^{\i\theta}}-\e^{i\theta}\frac{t^x}{(1+t\e^{\i\theta})^2}]) \\ &=
        \i\e^{\i x\theta}\xint_0^{+\infty}{h'(t)}
    \>
    où $h$ est la fonction indiquée dans l'énoncé :
    \< h(t)=\frac{t^x}{1+t \e^{\i\theta}} \>
    On a $h(t)\lsim_{t\to 0}t^x$, $h(t)\lsim_{t\to\infty}\xfrac{{\e^{-\i\theta}}}{{t^{1-x}}}$ et $x\in\ii]0,1[$
    \<\al{}
        \llim_{t\to0}h(t) &= 0 &&& \llim_{t\to+\infty}h(t)=0
    \>
    On en déduit que 
    \<
        g'(\theta)=\i\e^{\i\theta}\delim(\lim_{t\to+\infty}h(t)-\lim_{t\to 0}h(t))=0
    \>
    $g$ est de classe $\mathcal C^1$ de dérivée nulle sur \emph{l'intervalle} $\ii]-\pi,\pi[$ donc
    \<\r 
        La fonction $g$ est constante sur $\ii]-\pi,\pi[$.
    \>
    \begin{nb} 
        Une simple intégration par partie effectuée sur l'intégrale dans $r'(\theta)$ aboutit à la relation 
        \< r'(\theta)+\i xr(\theta)=0 \>
        exprimant ainsi que $g'(\theta)=0$. 
    \end{nb}
\end{solution}

\xques %4
    Montrer que pour tout $\theta \in\ii ] 0 , \pi[$,
    \<\compactbreak{
         g(\theta) \sin (x \theta)=\frac{1}{2 \i}\left(g(-\theta) \e^{\i x \theta}-g(\theta) \e^{-\i x \theta}\right)= {} \\
        \sin (\theta) \int_0^{+\infty} \frac{t^x}{t^2+2 t \cos (\theta)+1} \diff  t}
    \>
\begin{solution}
    Soit $\theta\in\ii]0,\pi[$. La fonction $g$ est constante et on a $\OV{g(\theta)}=g(-\theta)=g(\theta)$ donc 
    \begin{xalign}
        \<
            g(\theta)\sin(x\theta) &=
            \frac1{2\i}\delim(g(\theta)\e^{\i x\theta}-g(\theta)\e^{-\i x\theta}) 
        \>
        \eline 
        \<\lt{donc}\fr{result}
           \sff &=
            \frac1{2\i}\delim\sz2(g(-\theta)\e^{\i x\theta}-g(\theta)\e^{-\i x\theta})
        \>
        \eline 
        \<
            &=
            \xim\sz2(g(\theta)\e^{-\i x\theta})      
        \>
        (car $g(-\theta)=\OV{g(\theta)}$)
        \eline 
        \<
            &=
            \xim(\int_0^{+\infty}\frac{t^{x-1}}{1+t\e^{\i \theta}}\diff t)
        \>
        \eline 
        \<
            &=
            \xim(\int_0^{+\infty}\frac{t^{x-1}(1+t\e^{-i\theta})}{|1+t\e^{\i \theta}|^2}\diff t)
        \>
        \eline 
        \<\lt{d'où}\fr{result}
            \sff &=
            \sin(\theta)\int_0^{+\infty}\frac{t^x}{1+2t\cos\theta+t^2}\diff t
        \>
    \end{xalign}
    
\end{solution}

    
\xques %5
     En déduire que :
    \<
        \xforall \theta \in\ii ] 0 , \pi[\; 
        g(\theta) \sin (\theta x)=\int_{\xcotan \theta}^{+\infty} \frac{(u \sin (\theta)-\cos (\theta))^x}{1+u^2} \diff  u
    \>
    où  $\xcotan\theta)=\ds\frac{\cos (\theta)}{\sin (\theta)}$

\begin{solution}
Soit $\theta\in\ii]0,\pi[$. Sachant que $\sin\theta\ne0$ on peut poser pour tout réel $t>0$, $t=u\sin\theta-\cos\theta$. On a alors
    \<t^2+2t\cos\theta+1=(t+\cos\theta)^2+\sin^2\theta=(1+u^2)\sin^2\theta\>
L'application $u\longmapsto u\sin\theta-\cos\theta$ est une bijection de classe $\mathcal C^1$ de $\ii]\cotan\theta,+\infty[$ sur $\ii]0,+\infty[$ et l'integrale $\int_{0}^{+\infty}\frac{t^x}{1+2t\cos\theta+t^2}\diff t$ est convergente donc le changement de variable $t=u\sin\theta-\cos\theta$ donne 
\<
\int_{0}^{+\infty}\frac{t^x}{1+2t\cos\theta+t^2}\diff t=
\frac1{\sin\theta}\int_{\xcotan \theta}^{+\infty} \frac{(u \sin (\theta)-\cos (\theta))^x}{1+u^2} \diff  u
\>        
\<\lt{Soit}\fr{result}
    g(\theta) \sin (\theta x)=\int_{\xcotan \theta}^{+\infty} \frac{(u \sin (\theta)-\cos (\theta))^x}{1+u^2} \diff  u
\>
    
\end{solution}

\xques %6
Montrer, en utilisant le théorème de convergence dominée, que :
\<
\lim _{\theta \to \pi^{-}} g(\theta) \sin (x \theta)=\int_{-\infty}^{+\infty} \frac{\diff  u}{1+u^2}
\>

\begin{solution}
    Dans le but d'utiliser le \ac{tcd}, considérons une suite $(\theta_n)_n$ d'éléments de $\ii[0,\pi[$ qui converge ves $\pi$. On a alors pour tout $n\in\N$
    \<
        g(\theta_n)\sin(x\theta_n)=
        \int_{0}^{+\infty}\frac{t^x\sin(\theta_n)}{1+2t\cos\theta_n+t^2}\diff t=
        \int_{0}^{+\infty}h_n(t)\diff t  
    \>    
\end{solution}

\xques %7
 En déduire que
\<
\int_0^{+\infty} \frac{t^{x-1}}{1+t} \diff  t=\frac{\pi}{\sin (\pi x)}
\>
\exit 

\wparti{Une expression (utile) de la fonction sinus}
On rappelle que $x$ est un élément de $\ii ]0, 1[$ fixé.
\xques\r %8
 Montrer que.
\<
\int_0^{+\infty} \frac{t^{x-1}}{1+t} \diff  t=\int_0^1\left(\frac{t^{x-1}}{1+t}+\frac{t^{-x}}{1+t}\right) \diff  t
\>


\xques %9 
Montrer que:
\<
\int_1^1 \frac{t^{x-1}}{1+t} \diff  t=\sum_{k=0}^{+\infty} \frac{(-1)^k}{k+x}
\>

\xques %10 
 Établir l'identité
 \<
\int_0^{+\infty} \frac{t^{x-1}}{1+t} \diff  t=\sum_{n=0}^{+\infty} \frac{(-1)^n}{n+x}+\sum_{n=0}^{+\infty} \frac{(-1)^n}{n+1-x} .
\>
\xques %11  
En déduire que l'on a
\<
\frac{\pi}{\sin (\pi x)}=\frac{1}{x}-\sum_{n=1}^{+\infty} \frac{2(-1)^n x}{n^2-x^2} .
\>
\xques %12
 En déduire enfin que :
 \<
\forall y \in\ii ] 0 , \pi[, \quad \sum_{n=1}^{+\infty} \frac{2(-1)^n y \sin (y)}{y^2-n^2 \pi^2}=1-\frac{\sin (y)}{y}
\>
\exit 

\wparti{Calcul d'une intégrale de Dirichlet généralisée}
\xques\r %13
 Montrer que l'intégrale
 \<
\int_0^{+\infty} \frac{1-(\cos (t))^{2 p+1}}{t^2} \diff  t
\>
converge et que :
\<
\int_0^{+\infty} \frac{1-(\cos (t))^{2 p+1}}{t^2} \diff  t=(2 p+1) \int_0^{+\infty}(\cos (t))^{2 p} \frac{\sin (t)}{t} \diff  t .
\>
\xques %14
 Montrer que pour tout $n \in \N^*$ :
 \<
\int_{\frac{\pi}{2}+(n-1) \pi}^{\frac{\pi}{2}+n \pi}(\cos (t))^{2 p} \frac{\sin (t)}{t} \diff  t=\int_0^{\frac{\pi}{2}}(\cos (t))^{2 p} \frac{2(-1)^n t \sin (t)}{t^2-n^2 \pi^2} \diff  t .
\>
\xques %15
 En déduire que :
 \<
\int_{\frac{\pi}{2}}^{+\infty}(\cos (t))^{2 p} \frac{\sin (t)}{t} \diff  t=\int_0^{\frac{\pi}{2}}(\cos (t))^{2 p}\left(\sum_{n=1}^{+\infty} \frac{2(-1)^n t \sin (t)}{t^2-n^2 \pi^2}\right) \diff  t .
\>
\xques %16
 En déduire que :
 \<
\int_0^{+\infty}(\cos (t))^{2 p} \frac{\sin (t)}{t} \diff t=\int_0^{\frac{\pi}{2}}(\cos (t))^{2 p} \diff  t .
\>

Dans le cas $p=0$, cette intégrale est communément appelée "Intégrale de Dirichlet".

\xques %17
 Montrer que :
 \<
(\cos (t))^{2 p}=\frac{1}{2^{2 p}}\delim(\binom{2 p}{p}+2 \sum_{k=0}^{p-1}\binom{2 p}{k} \cos (2(p-k) t))
\>

\begin{ind}
On pourra développer $\ds\left(\frac{\e^{\i t}+\e^{-\i t}}{2}\right)^{2 p}$.
\end{ind}
\xques %18
 En déduire que :
 \<
\int_0^{+\infty} \frac{1-(\cos (t))^{2 p+1}}{t^2} \diff  t=\frac{\pi}{2} \frac{(2 p+1)!}{2^{2 p} \cdot(p!)^2}
\>
\exit

\wparti{Calcul de \(E(|S_n|)\)}
Toutes les variables aléatoires sont définies sur un même espace probabilisé $(\Omega, \mathcal{A}, P)$. Soient $\left(X_k\right)_{k \in\N^*}$ des variables aléatoires indépendantes, de même loi donnée par :
\<
\xPr(X_1=-1)=
\xPr(X_1=1)=\frac{1}{2}
\>


Pour tout $n \in \N^*$, on note $S_n=\sum_{k=1}^n X_k$.

\xques\r %19 
Déterminer, pour tout $n \in \N^*, \xEs(S_n)$ et 
$\xVa(S_n)$.

Soient $S$ et $T$ deux variables aléatoires indépendantes prenant toutes deux un nombre fini de valeurs réelles. On suppose que $T$ et $-T$ suivent la même loi.
\xques %20
 Montrer que :
 \<
\xEs\sz1(\cos (S+T))=\xEs\sz1(\cos (S)) \xEs(\cos (T)) .
\>
\xques %21
 En déduire que pour tout $n \in \N^*$, et pour tout $t \in \R $ :
 \<
\xEs(\cos \left(t S_n\right))=(\cos (t))^n
\>

\xques %22
Soient $a,b\in\R$ tels que $a\ne 0$ et $|b|\leq |a|$. Montrer que 
\< |a+b|=|a|+\xsign(a)b \>
ou $\sign(a)=\xfrac{x}{{|x|}}$ pour $x$ réel non nul- En déduire que
\< 
    \xforall n\in\N^*\;
    \xEs(|S_{2n}|)=\xEs(|S_{2n-1|})
\>

\xques %23
Montrer que pour tout $s\in\R$
\< \int_0^{+\infty}\frac{1-\cos(st)}{t^2}\diff t =\frac\pi2|s|\>

\xques %24
En déduire que pour tout $n\in\N^*$ :
\< \xEs(|S_n|)=\frac2\pi
\int_0^{+\infty}\frac{1-\delim\sz1(\cos t)^n}{t^2}\diff t
\>

\xques %25
Conclure que :
\<
    \xforall n\in\N^*\;
    \xEs(|S_{2n}|)=\xEs(|S_{2n-1}|)=
    \frac{(2n-1)!}{2^{2n-1}\delim\sz2((n-1)!)^2}
\>
\exit 
\end{enonce} 

\end{document}